\let\negmedspace\undefined
\let\negthickspace\undefined
\documentclass[journal]{IEEEtran}
\usepackage[a5paper, margin=10mm, onecolumn]{geometry}
%\usepackage{lmodern} % Ensure lmodern is loaded for pdflatex
\usepackage{tfrupee} % Include tfrupee package

\setlength{\headheight}{1cm} % Set the height of the header box
\setlength{\headsep}{0mm}     % Set the distance between the header box and the top of the text

\usepackage{gvv-book}
\usepackage{gvv}
\usepackage{cite}
\usepackage{amsmath,amssymb,amsfonts,amsthm}
\usepackage{algorithmic}
\usepackage{graphicx}
\usepackage{textcomp}
\usepackage{xcolor}
\usepackage{txfonts}
\usepackage{listings}
\usepackage{enumitem}
\usepackage{mathtools}
\usepackage{gensymb}
\usepackage{comment}
\usepackage[breaklinks=true]{hyperref}
\usepackage{tkz-euclide} 
\usepackage{listings}
% \usepackage{gvv}                                        
\def\inputGnumericTable{}                                 
\usepackage[latin1]{inputenc}                                
\usepackage{color}                                            
\usepackage{array}                                            
\usepackage{longtable}                                       
\usepackage{calc}                                             
\usepackage{multirow}                                         
\usepackage{hhline}                                           
\usepackage{ifthen}                                           
\usepackage{lscape}
\begin{document}

\bibliographystyle{IEEEtran}
\vspace{3cm}

\title{10.4.1.1.2}
\author{EE24BTECH11019 - Dwarak A}
% \maketitle
% \newpage
% \bigskip
{\let\newpage\relax\maketitle}

\renewcommand{\thefigure}{\theenumi}
\renewcommand{\thetable}{\theenumi}
\setlength{\intextsep}{10pt} % Space between text and floats


\numberwithin{equation}{enumi}
\numberwithin{figure}{enumi}
\renewcommand{\thetable}{\theenumi}

\textbf{Question:}

Find the roots of the quadratic equation:
\begin{align}
    x^2 - 2x = \brak{-2}\brak{3-x}
\end{align}

\solution

Rearranging terms,
\begin{align}
    x^2 - 2x &= 2x - 6 \\
    x^2 - 4x + 6 &= 0
\end{align}

Theoretical solution (Quadratic formula):

The roots are,
\begin{align}
    x_1 &= \frac{-b+\sqrt{b^2 - 4ac}}{2a} \\
    &= \frac{4+\sqrt{16-24}}{2} \\
    &= 2+\sqrt{2}i \\
    x_2 &= \frac{-b-\sqrt{b^2 - 4ac}}{2a} \\
    &= \frac{4-\sqrt{16-24}}{2} \\
    &= 2-\sqrt{2}i
\end{align}

Computational solution:

(1) Eigenvalues of Companion Matrix:

The roots of a polynomial equation $x^n+b_{n-1}x^{n-1}+\dots+b_2x^2+b_1x+b_0 = 0$ is given by finding eigenvalues of the companion matrix \brak{C}.
\begin{align}
    C = \myvec{0&1&0&\dots&0\\ 0&0&1&\dots&0\\ \vdots &\vdots &\vdots &\ddots&\vdots\\0&0&0&\vdots&1\\-b_0&-b_1&-b_2&\dots&-b_{n-1}}
\end{align}

The solution given by the code is,
\begin{align}
    x_1 &= 2.00000000+1.41421356i \\
    x_2 &= 2.00000000-1.41421356j
\end{align}

(2) Newton-Raphson iterative method:
\begin{align}
    f\brak{x} &= x^2 - 4x + 6 \\
    f^\prime\brak{x} &= 2x - 4
\end{align}

Difference equation,
\begin{align}
    x_{n+1} &= x_n - \frac{f\brak{x_{n}}}{f^\prime\brak{x_n}} \\
    x_{n+1} &= x_n - \frac{x_n^2 - 4x_n + 6}{2x_n - 4} \\
    x_{n+1} &= \frac{x_n}{2} - 1 + \frac{1}{x_n - 2}
\end{align}

Picking two initial guesses,
\begin{align}
    x_0 &= 1+i \text{ converges to } 2.0 + 1.4142135623730954i \\
    x_0 &= -1-i \text{ converges to } 2.0000000000000733 + -1.4142135623729934i
\end{align}

\end{document}}
